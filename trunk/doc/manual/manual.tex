%%%%%%%%%%%%%%%%%%%%%%%%%%%%%%%%%%%%%%%%%%%%%%%%%%%%%%%%%%%%%%%%%%%%%%%%%%%%
% Latex source file for the LCCD users manual
% the file can be processed with pdflatex and latex2html(using .jpg and .pdg) 
% and plain latex (using .eps) 
%
%  --- to create  an html file  this manual :
% latex2html -mkdir -dir html -info 0 -no_auto_link -split 0 -no_navigation manual.tex 
% ----- or pdf:
% pdflatex manual.tex
% ----- or postscript:
% latex manual.tex ; dvips manual.dvi
%
%%%%%%%%%%%%%%%%%%%%%%%%%%%%%%%%%%%%%%%%%%%%%%%%%%%%%%%%%%%%%%%%%%%%%%%%%%%%
\documentclass[twoside]{article}
%\documentclass[twoside]{report}
% --- the following allows to use either latex/latex2html or pdf latex
% --- with the same file and have the right images included (eps or jpg/png)
\newif\ifpdf\ifx\pdfoutput\undefined\pdffalse\else\pdfoutput=1\pdftrue\fi
\newcommand{\pdfgraphics}{\ifpdf\DeclareGraphicsExtensions{.png,.jpg}\fi}
%-------------------------------------------------------------------------
\usepackage{graphicx}
%\usepackage{fancyhdr}
\usepackage{verbatim}
\usepackage{html}
%\pagestyle{fancy}
%\fancyhead{} % clear all fields
%\fancyhead[C]{\it {LCCD - Users manual}}
%\fancyhead[RO,LE]{\thepage}
%\fancyfoot{} % clear all fields
%\renewcommand{\headrulewidth}{0pt}
%\renewcommand{\footrulewidth}{0pt}
\renewcommand{\appendixname}{Appendix}

\newcommand{\captionstyle}[1]{\textit{\normal{#1}}}

\newcommand{\Href}[1]{\htmladdnormallink{\footnotesize {#1}}{#1}}

\setlength{\textheight}{235mm}
\setlength{\textwidth}{155mm}
\setlength{\topmargin}{-20mm}
\setlength{\evensidemargin}{10mm}
\setlength{\oddsidemargin}{10mm}

%\setlength{\parskip}{\baselineskip}
\setlength{\parindent}{0pt}

\bibliographystyle{apsrev}

\begin{document}
\pdfgraphics

\title{{\Huge\bf LCCD -  Users manual} \\ v00-01}

\author{F. Gaede \\  DESY}


\maketitle
\setcounter{tocdepth}{2}
\tableofcontents

%\thispagestyle{fancy}

\begin{abstract}
Upcoming testbeam efforts will need a way to store and retrieve conditions data, e.g. 
slow control, electronics setup and calibration constants. 
Typically experiments use a 'conditions database' for this purpose. 
While a database offers you all the functionality that is required 
(varying validity time ranges, tags, history, etc.) it also puts some burden on the users
as they have to set up and maintain a data base system. 
A simpler approach is to store conditions data in LCIO~\cite{lcio_home} files that are used for the data itself 
anyhow.
While being straight forward and easy to implement this approach lacks some of the desired
features  for conditions data  like tags, versioning and history. 

LCCD is a toolkit that combines the two options in a transparent way.
It is implemented in C++ and uses an Open Source implementation of a conditions database 
interface developed for the Atlas experiment: ConditionsDBMySQL~\cite{conddb_home}

%<a href="http://conditionsdb.cvs.cern.ch/cgi-bin/conditionsdb.cgi/conddb/CondDBMySQL">
%(cvs)</a> .
\end{abstract}

\newpage
\section{INTRODUCTION \label{intro}}

{\Large To Be Written ...}
%\begin{figure}
%\includegraphics[width=150mm]{packages}    
%\caption{\captionstyle{Overview of the package dependencies for LCCD. }}
%\end{figure}

\section{Installation}
{\Large To Be Written ...}

\section{Using LCCD}

{\Large To Be Written ...}


\clearpage

%\section{SIO File Format of LCCD} \label{app_lcio}
%The following is a detailed xml-based description of the currently used SIO files.
%It describes the  detailed layout of the peristent data in the files 
%and thus can also be used as a reference of the data model used in LCCD.
%{\footnotesize \verbatiminput{../../../doc/lcio.xml} }
%\newpage 



\newpage
% Create the reference section using BibTeX:
%\bibliography{basename of .bib file}

\begin{thebibliography}{9}   % Use for  1-9  references
  %\begin{thebibliography}{99} % Use for 10-99 references
\bibitem{lcio_home}
  LCIO Homepage: \\
  \Href{http://lcio.desy.de}
\bibitem{lccd_home}
  LCCD Homepage: \\
  \Href{http://www.desy.de/~gaede/lccd}
\bibitem{marlin_home}
  Marlin Homepage: \\
  \Href{http://www.desy.de/~gaede/marlin}
\bibitem{conddb_home}
  ConditionsDBMySQL Homepage: \\
\Href{http://savannah.cern.ch/projects/conddb-mysql}  
\end{thebibliography}


\end{document}

